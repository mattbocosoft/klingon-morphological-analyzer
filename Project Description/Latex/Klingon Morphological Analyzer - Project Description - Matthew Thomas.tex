%
% File acl2014.tex
%
% Contact: koller@ling.uni-potsdam.de, yusuke@nii.ac.jp
%%
%% Based on the style files for ACL-2013, which were, in turn,
%% Based on the style files for ACL-2012, which were, in turn,
%% based on the style files for ACL-2011, which were, in turn, 
%% based on the style files for ACL-2010, which were, in turn, 
%% based on the style files for ACL-IJCNLP-2009, which were, in turn,
%% based on the style files for EACL-2009 and IJCNLP-2008...

%% Based on the style files for EACL 2006 by 
%%e.agirre@ehu.es or Sergi.Balari@uab.es
%% and that of ACL 08 by Joakim Nivre and Noah Smith

\documentclass[11pt]{article}
\usepackage{acl2014}
\usepackage{times}
\usepackage{url}
\usepackage{latexsym}
\usepackage{color}
\usepackage[dvipsnames]{xcolor}

%\setlength\titlebox{5cm}

% You can expand the titlebox if you need extra space
% to show all the authors. Please do not make the titlebox
% smaller than 5cm (the original size); we will check this
% in the camera-ready version and ask you to change it back.


\title{A finite-state morphological analyzer using foma for Klingon, a constructed language}

\author{Matthew David Dreselly Thomas \\
  University of Colorado at Boulder \\
  {\tt matt@bocosoft.net} \\}

\date{}

\begin{document}
\maketitle
\begin{abstract}
This paper presents a morphological analyzer for Klingon, constructed language, developed following the official description of the morphology in the Klingon Dictionary~\cite{Okrand:92}.

The open-source tool \textit{foma} for constructing finite-state automata and transducers is used for building the morphological analyzer.

\end{abstract}

\section{Introduction}

Klingon was created by Marc Okrand, an American Linguist, for the 1984 movie Star Trek III. Klingon is spoken by the fictional Klingons in the Star Trek universe, so Okrand designed the Klingon language to sound alien by using grammatical constructs and sound combinations infrequently used in natural language~\cite{KLI:Development,KLI:Sounds}.

Klingon has a complete grammar that can be spoken and used as a real language. Marc Okrand formalized the Klingon grammar and vocabulary in The Klingon Dictionary~\cite{Okrand:92}.

The number of fluent Klingon speakers is not known, but has been estimated between a few dozen~\cite{Okrent:09} to around 100~\cite{Kelly:13}. The Klingon Dictionary~\cite{Okrand:92} has sold more than 300,000 copies and has been translated into 4 languages. Klingon has entered into popular culture, and literary works such Shakespeare's Hamlet, Gilgamesh, and "A Christmas Carol" have been translated into Klingon. The Bible is currently being translated~\cite{KLI:Bible} but is a difficult task due to limited Klingon vocabulary. The current lexicon contains almost 3,000 words~\cite{Zrajm:12} and only grammar and words coined by Marc Okrand are considered canonical by the Klingon speaking community~\cite{KLI:Canonical}.

\section{Klingon Morphology}

Klingon is an agglutinative language that uses Object-Verb-Subject word order. Okrand categorizes down the Klingon word types into Nouns, Verbs, and Other.

\subsection {Nouns}

Nouns can take five types of suffixes. Each suffix type, if present, must be appended to the noun stem in a specific order and there can be no more than one suffix per type. The suffixes that a noun can take also depend on the type of noun. Nouns are split into three types; body-parts, beings capable of language, and general nouns.

\begin{description}
	\item[Type 1] Augmentative/diminutive
	\item[Type 2] Number

	\begin{center}
	\begin{tabular}{ | l | l | l | l |}
	\hline
	\bf{Noun Type} & \bf{Suffix} \\ \hline
	Body-Part & -pu' \\ \hline
	Language-Capable & -Du' \\ \hline
	Generic & -mey \\
	\hline
	\end{tabular}
	\end{center}
	
	The \textit{-mey} suffix may also be used with body-part and beings-capable-of-language nouns, but the resulting word acquires an additional meaning "\textit{all over the place}" on top of the plural meaning.
	
\begin{table}[h]
\begin{center}
\begin{tabular}{l|l}
\bf Klingon & \bf English \\
jup & friend \\
jupDu' & friends \\
jupmey & friends all over the place \\
\end{tabular}
\end{center}
\caption{Generic Plural Suffix Usage with Inherently Singular Nouns}
\end{table}

	There are some nouns that are inherently singular or inherently plural. For example the Klingon word \textit{peng} means torpedo (singular) but \textit{cha} means torpedos (plural). The singular forms of these types of nouns may also take the \textit{-mey} generic plural suffix, but the resulting word also acquires the "\textit{all over the place}" meaning.
	
	\begin{table}[h]
\begin{center}
\begin{tabular}{l|l}
\bf Klingon & \bf English \\
peng & torpedo \\
pengmey & torpedos all over the place \\
cha & torpedos \\
**chamey & \it Not allowed \\
\end{tabular}
\end{center}
\caption{Generic Plural Suffix Usage with Language-Capable Nouns}
\end{table}

	Lastly, nouns in Klingon are not required to take any plural suffix in order to be plural. The singular form of a noun may be used where the plural form would normally be used, and this is perfectly grammatically correct. The meaning is disambiguated using the context of the word in the sentence. Therefore \textit{jup} means both "friend" and "friends" depending on the context.

	\item[Type 3] Qualification
	\item[Type 4] Possession/specification
	
	Possession suffixes are selected depending on whether the \textit{object} being referred to is language-capable.

	\begin{center}
	\begin{tabular}{ | l | l | l | l |}
	\hline
	\bf{Possessive} & \bf{Generic} & \bf{Language} \\
	\bf{Suffixes} & & \bf{Capable} \\ \hline
	my & -wIj & -wI' \\ \hline
	your & -lIj & -lI' \\ \hline
	his, her, its & -Daj & \textcolor{gray}{-Daj} \\ \hline
	our & -maj & -ma' \\ \hline
	your (plural) & -raj & -ra' \\ \hline
	their & -chaj & \textcolor{gray}{-chaj} \\
	\hline
	\end{tabular}
	\end{center}

	\item[Type 5] Syntactic markers
\end{description}

\subsection {Verbs}

Verb can take both prefixes and suffixes. Prefixes specify both the subject and object of the verb, including whether or not a subject and/or object exists. Verbs can use either the \textbf{Indicative mood} or the \textbf{Imperative mood}, which can also affect whether some suffixes can be used. There is no infinite form of a verb in Klingon.

Verbs are categorized into \textbf{stative} and \textbf{dynamic}. Klingon does not use adjectives so stative verbs are used to describe nouns. For example to say "red ship", one would use the stative verb Doq meaning to be red/orange, "\textit{Duj Doq}".

Verbs conform to nine ordered suffix types. Just like nouns, each suffix type, if present, must be appended to the verb stem in a specific order. There can be no more than one suffix per type.

\begin{description}
	\item[Type 1] Reflexive, Reciprocal
	\item[Type 2] Volition, Necessity
	\item[Type 3] Inceptive, Inchoative
	\item[Type 4] Causative
	\item[Type 5] Undefined Subject, Capability
	\item[Type 6] Perfection, Uncertainty
	\item[Type 7] Aspect
	\item[Type 8] Honorific
	\item[Type 9] Syntactic markers
\end{description}

In addition, a "rover" suffix type can be appended immediately after the verb stem or between the nine suffix types, however rover suffixes cannot be appended after a Type-9 suffix. There are no restrictions on the number of rover suffixes that may be used. A rover suffix's meaning applies to the suffix immediately proceeding it. For example the \textbf{negating} rover suffix \textit{be'} can be appended immediately after the verb stem and/or after any other suffix type.

\begin{table}[h]
\begin{center}
\begin{tabular}{l|l}
\bf Klingon & \bf English \\
Dub & to improve \\
Dubbe' & to not improve \\
DubmoH & to cause to improve \\
DubmoHbe' & to not cause to improve \\
Dubbe'moH & to cause to not improve \\
Dubbe'moHbe' & to not cause to not improve
\end{tabular}
\end{center}
\caption{Verb Rover Suffix Usage }
\end{table}

Rover suffixes can even be appended after other rover suffixes, e.g. \textit{Dubbe'be'} meaning \textit{to not not improve}.

\subsection {Other Words}

Pronouns

Numbers

Conjunctions

Adverbials

Exclamations

Names

Questions

\section{Klingon Morphophonology}

Klingon completely lacks morphophonology with one exception. The endearment noun suffix \textit{-oy} is prepended with a glottal stop \textit{'} if the noun stem ends with a vowel.

\begin{table}[h]
\begin{center}
\begin{tabular}{|l|l|}
\hline \bf Noun Stem & \bf Noun with Endearment Suffix -oy \\ \hline
jup (friend) & jupoy (dear friend) \\
cha (torpedo) & cha'oy (dear torpedo) \\
\hline
\end{tabular}
\end{center}
\caption{Endearment Suffix }
\end{table}

\section{FST Implementation}

Used lexc file.
Used foma file for rewrite rule(s).

% include your own bib file like this:
%\bibliographystyle{acl}
%\bibliography{acl2014}

\begin{thebibliography}{}

\bibitem[\protect\citename{Okrand}1992]{Okrand:92}
Marc Okrand.
\newblock 1992.
\newblock {\em The Klingon Dictionary}, 2nd edition.
\newblock New York: Pocket Books, Simon and Schuster Inc.

\bibitem[\protect\citename{KLI}Development]{KLI:Development}
\newblock {\em Development and Use of the Klingon Language.}
\newblock The Klingon Language Institute
\newblock \url{http://www.kli.org/about-klingon/klingon-history/}

\bibitem[\protect\citename{KLI}Sounds]{KLI:Sounds}
\newblock {\em The Sounds of Klingon.}
\newblock The Klingon Language Institute
\newblock \url{http://www.speakklingon.info/about-klingon/sounds-of-klingon/}

\bibitem[\protect\citename{KLI}Canonical]{KLI:Canonical}
\newblock {\em Canonical Sources.}
\newblock The Klingon Language Institute
\newblock \url{http://www.kli.org/wiki/Canonical_sources}

\bibitem[\protect\citename{KLI}Bible]{KLI:Bible}
\newblock {\em Klingon Bible Translation Project.}
\newblock The Klingon Language Institute
\newblock \url{http://www.kli.org/wiki/Klingon_Bible_Translation_Project}

\bibitem[\protect\citename{Okrent}2009]{Okrent:09}
Arika Okrent.
\newblock 2009.
\newblock {\em In the Land of Invented Languages}, 1st edition.
\newblock New York: Spiegel \& Grau

\bibitem[\protect\citename{Zrajm}2012]{Zrajm:12}
Zrajm.
\newblock 2012.
\newblock {\em Klingon Word Database}, 1st edition.
\newblock Klingonska Akademien

\bibitem[\protect\citename{Kelly}2013]{Kelly:13}
Samantha Murphy Kelly.
\newblock 2013.
\newblock {\em How Klingon Became a Universal Language}
\newblock Mashable.com
\newblock \url{http://mashable.com/2013/05/18/klingon-star-trek/}

\end{thebibliography}

\section{Examples of text}

\textbf{The proceedings are designed for printing on A4
  paper.}
\texttt{yusuke@nii.ac.jp}
\textit{pdflatex}
\texttt{ dvips}

\begin{quote}
Hello there
\end{quote}

\end{document}
